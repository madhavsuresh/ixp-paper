\documentclass{acm_proc_article-sp}

\begin{document}

\title{Determining Remote Peerings at IXPs}
%
% You need the command \numberofauthors to handle the 'placement
% and alignment' of the authors beneath the title.
%
% For aesthetic reasons, we recommend 'three authors at a time'
% i.e. three 'name/affiliation blocks' be placed beneath the title.
%
% NOTE: You are NOT restricted in how many 'rows' of
% "name/affiliations" may appear. We just ask that you restrict
% the number of 'columns' to three.
%
% Because of the available 'opening page real-estate'
% we ask you to refrain from putting more than six authors
% (two rows with three columns) beneath the article title.
% More than six makes the first-page appear very cluttered indeed.
%
% Use the \alignauthor commands to handle the names
% and affiliations for an 'aesthetic maximum' of six authors.
% Add names, affiliations, addresses for
% the seventh etc. author(s) as the argument for the
% \additionalauthors command.
% These 'additional authors' will be output/set for you
% without further effort on your part as the last section in
% the body of your article BEFORE References or any Appendices.

\numberofauthors{2} %  in this sample file, there are a *total*
% of EIGHT authors. SIX appear on the 'first-page' (for formatting
% reasons) and the remaining two appear in the \additionalauthors section.
%
\author{
% You can go ahead and credit any number of authors here,
% e.g. one 'row of three' or two rows (consisting of one row of three
% and a second row of one, two or three).
%
% The command \alignauthor (no curly braces needed) should
% precede each author name, affiliation/snail-mail address and
% e-mail address. Additionally, tag each line of
% affiliation/address with \affaddr, and tag the
% e-mail address with \email.
%
% 1st. author
\alignauthor
Mario Sanchez\\
       \affaddr{Northwestern University}\\
       \affaddr{EECS Department}\\
       \affaddr{Evanston, IL}\\
       \email{msanchez@northwestern.edu}
% 2nd. author
\alignauthor
Madhav Suresh\\
       \affaddr{Northwestern University}\\
       \affaddr{EECS Department}\\
       \affaddr{Evanston, IL}\\
       \email{madhav@u.northwestern.edu}
}
\date{16 March 2012}
% Just remember to make sure that the TOTAL number of authors
% is the number that will appear on the first page PLUS the
% number that will appear in the \additionalauthors section.

\maketitle
\begin{abstract}
Peering matrices of Internet exchange points (IXPs) have been an area of interest because of their
critical role in the flow of traffic. \cite{Augustin:2009}
Remote peering at IXPs has been a recent trend, as it allows smaller ASs to bypass more costly 
peering agreements.
Efforts have been made to discover peering matrices at IXPs however no work has been done 
to discover remote peers. We designed and implemented a tool that can determine remote peerings
given a peering list of an IXP. Using traceroutes obtained from Dasu \cite{Sanchez:2011}, geolocation
techniques and reverse dns, we were able to determine remote peerings with 
%TODO: 
(WHAT?) level of accuracy.
\end{abstract}

% A category with the (minimum) three required fields
%\category{H.4}{Information Systems Applications}{Miscellaneous}
%A category including the fourth, optional field follows...
%\category{D.2.8}{Software Engineering}{Metrics}[complexity measures, performance measures]

%\terms{IXP, remote peering}

%\keywords{ACM proceedings, \LaTeX, text tagging} % NOT required for Proceedings

\section{Introduction}

\section{System Design}


\section{Usage}

\section{Approach}
\label{sec:approach}

\section{Future Work}
\label{sec:future}
Much to do.

\bibliographystyle{abbrv}
\bibliography{ixp}

\end{document}
